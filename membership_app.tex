\documentclass[a4paper,10pt,BCOR10mm,oneside,headsepline]{scrartcl}
\usepackage[ngerman]{babel}
\usepackage[utf8]{inputenc}
\usepackage{wasysym}% provides \ocircle and \Box
\usepackage{enumitem}% easy control of topsep and leftmargin for lists
\usepackage{color}% used for background color
\usepackage{forloop}% used for \Qrating and \Qlines
\usepackage{ifthen}% used for \Qitem and \QItem
\usepackage{typearea}
\areaset{17cm}{26cm}
\setlength{\topmargin}{-1cm}
\usepackage{scrpage2}
\pagestyle{scrheadings}
\ihead{Wayne State University Linux Users Group Membership Application}
\ohead{\pagemark}
\chead{}
\cfoot{WSULUG Membership Application 2012-08-27}

%%%%%%%%%%%%%%%%%%%%%%%%%%%%%%%%%%%%%%%%%%%%%%%%%%%%%%%%%%%%
%% Beginning of questionnaire command definitions         %%
%%%%%%%%%%%%%%%%%%%%%%%%%%%%%%%%%%%%%%%%%%%%%%%%%%%%%%%%%%%%
%%
%% 2010 by Sven Hartenstein
%% mail@svenhartenstein.de
%% http://www.svenhartenstein.de
%%
%% Please be warned that this is NOT a full-featured framework for
%% creating (all sorts of) questionnaires. Rather, it is a small
%% collection of LaTeX commands that I found useful when creating a
%% questionnaire. Feel free to copy and adjust any parts you like.
%% Most probably, you will want to change the commands, so that they
%% fit your taste.
%%
%% Also note that I am not a LaTeX expert! Things can very likely be
%% done much more elegant than I was able to. If you have suggestions
%% about what can be improved please send me an email. I intend to
%% add good tipps to my website and to name contributers of course.

%% \Qq = Questionaire question. Oh, this is just too simple. It helps
%% making it easy to globally change the appearance of questions.
\newcommand{\Qq}[1]{\textbf{#1}}

%% \QO = Circle or box to be ticked. Used both by direct call and by
%% \Qrating and \Qlist.
\newcommand{\QO}{$\Box$}% or: $\ocircle$

%% \Qrating = Automatically create a rating scale with NUM steps, like
%% this: 0--0--0--0--0.
\newcounter{qr}
\newcommand{\Qrating}[1]{\QO\forloop{qr}{1}{\value{qr} < #1}{---\QO}}

%% \Qline = Again, this is very simple. It helps setting the line
%% thickness globally. Used both by direct call and by \Qlines.
\newcommand{\Qline}[1]{\rule{#1}{0.6pt}}

%% \Qlines = Insert NUM lines with width=\linewith. You can change the
%% \vskip value to adjust the spacing.
\newcounter{ql}
\newcommand{\Qlines}[1]{\forloop{ql}{0}{\value{ql}<#1}{\vskip0em\Qline{\linewidth}}}

%% \Qlist = This is an environment very similar to itemize but with
%% \QO in front of each list item. Useful for classical multiple
%% choice. Change leftmargin and topsep accourding to your taste.
\newenvironment{Qlist}{%
  \renewcommand{\labelitemi}{\QO}
  \begin{itemize}[leftmargin=1.5em,topsep=-.5em]
}{%
  \end{itemize}
}

%% \Qtab = A "tabulator simulation". The first argument is the
%% distance from the left margin. The second argument is content which
%% is indented within the current row.
\newlength{\qt}
\newcommand{\Qtab}[2]{
  \setlength{\qt}{\linewidth}
  \addtolength{\qt}{-#1}
  \hfill\parbox[t]{\qt}{\raggedright #2}
}

%% \Qitem = Item with automatic numbering. The first optional argument
%% can be used to create sub-items like 2a, 2b, 2c, ... The item
%% number is increased if the first argument is omitted or equals 'a'.
%% You will have to adjust this if you prefer a different numbering
%% scheme. Adjust topsep and leftmargin as needed.
\newcounter{itemnummer}
\newcommand{\Qitem}[2][]{% #1 optional, #2 notwendig
  \ifthenelse{\equal{#1}{}}{\stepcounter{itemnummer}}{}
  \ifthenelse{\equal{#1}{a}}{\stepcounter{itemnummer}}{}
  \begin{enumerate}[topsep=2pt,leftmargin=2.8em]
  \item[\textbf{\arabic{itemnummer}#1.}] #2
  \end{enumerate}
}

%% \QItem = Like \Qitem but with alternating background color. This
%% might be error prone as I hard-coded some lengths (-5.25pt and
%% -3pt)! I do not yet understand why I need them.
\definecolor{bgodd}{rgb}{0.8,0.8,0.8}
\definecolor{bgeven}{rgb}{0.9,0.9,0.9}
\newcounter{itemoddeven}
\newlength{\gb}
\newcommand{\QItem}[2][]{% #1 optional, #2 notwendig
  \setlength{\gb}{\linewidth}
  \addtolength{\gb}{-5.25pt}
  \ifthenelse{\equal{\value{itemoddeven}}{0}}{%
    \noindent\colorbox{bgeven}{\hskip-3pt\begin{minipage}{\gb}\Qitem[#1]{#2}\end{minipage}}%
    \stepcounter{itemoddeven}%
  }{%
    \noindent\colorbox{bgodd}{\hskip-3pt\begin{minipage}{\gb}\Qitem[#1]{#2}\end{minipage}}%
    \setcounter{itemoddeven}{0}%
  }
}

%%%%%%%%%%%%%%%%%%%%%%%%%%%%%%%%%%%%%%%%%%%%%%%%%%%%%%%%%%%%
%% End of questionnaire command definitions               %%
%%%%%%%%%%%%%%%%%%%%%%%%%%%%%%%%%%%%%%%%%%%%%%%%%%%%%%%%%%%%

\begin{document}

\renewcommand{\QO}{$\ocircle$}% Use circles now instead of boxes.

\begin{center}
  \textbf{\huge WSULUG Membership Application}
\end{center}\vskip1em

\noindent You've chosen to join the Wayne State University Linux Users Group. Congratulations, this may be the best decision you've ever made. Before it's official, though, please fill out this form to let us know what sorts of thing you know about, what sorts of things you'd like to learn about, and what sorts of things you'd like to learn about. We'll use this information from all members to better tailor group meetings and activities to the membership.

If you need any more room to write, feel free to write on the back of this paper.

\noindent \textit{Please note there is absolutely no wrong answer here - if you've barely even heard of Linux, and just are curious to learn about it, that's absolutely fine.}


\section*{About you}

\Qitem{ \Qq{Your name}: \Qline{8cm} }

\Qitem{ \Qq{AccessID}: \Qline{8cm} }

\Qitem{ \Qq{Email} (if not accessid@wayne.edu): \Qline{8cm} }

\Qitem{ \Qq{When works best for you to have meetings?} \Qlines{2} }

\Qitem{ \Qq{How are you involved with Wayne State University?}
  \begin{Qlist}
  \item Student
  \item Employee
  \item Faculty
  \item Alumn\{{us,a\}}
  \item Some other way
  \item I'm not
  \end{Qlist}
}

\minisec{If you're a student...}

\Qitem{ \Qq{Do you have a major (and what, if so)?} \Qline{6cm}}

\Qitem{ \Qq{How long have you been at WSU?}
  \begin{Qlist}
  \item $<$ 1 year
  \item 1 - 2 years
  \item 2 - 3 years
  \item 3 - 4 years
  \item Over 4 years
  \end{Qlist}
}

\Qitem{ \Qq{Which are you?}
  \begin{Qlist}
  \item Undergrad
  \item Graduate Student
  \item Something else
  \end{Qlist}
}
\clearpage


\section*{About you and Linux}

\minisec{Please rate your interest in these areas}
\QItem[a]{ \Qq{Security} \Qtab{5cm}{non-existent \Qrating{10} strong}}

\QItem[b]{ \Qq{System Admin} \Qtab{5cm}{non-existent \Qrating{10} strong}}

\QItem[c]{ \Qq{Programming} \Qtab{5cm}{non-existent \Qrating{10} strong}}

\QItem[d]{ \Qq{Desktop Use} \Qtab{5cm}{non-existent \Qrating{10} strong}}

\QItem[e]{ \Qq{Scientific Computing} \Qtab{5cm}{non-existent \Qrating{10} strong}}

\QItem[f]{ \Qq{Networking} \Qtab{5cm}{non-existent \Qrating{10} strong}}

\QItem[g]{ \Qq{Databases} \Qtab{5cm}{non-existent \Qrating{10} strong}}

\QItem[h]{ \Qq{Licensing/Law} \Qtab{5cm}{non-existent \Qrating{10} strong}}
\vspace{.5cm}

\minisec{Please rate your knowledge in these areas}
\QItem[a]{ \Qq{Security} \Qtab{5cm}{novice \Qrating{10} expert}}

\QItem[b]{ \Qq{System Admin} \Qtab{5cm}{novice \Qrating{10} expert}}

\QItem[c]{ \Qq{Programming} \Qtab{5cm}{novice \Qrating{10} expert}}

\QItem[d]{ \Qq{Desktop Use} \Qtab{5cm}{novice \Qrating{10} expert}}

\QItem[e]{ \Qq{Scientific Computing} \Qtab{5cm}{novice \Qrating{10} expert}}

\QItem[f]{ \Qq{Networking} \Qtab{5cm}{novice \Qrating{10} expert}}

\QItem[g]{ \Qq{Databases} \Qtab{5cm}{novice \Qrating{10} expert}}

\QItem[h]{ \Qq{Licensing/Law} \Qtab{5cm}{novice \Qrating{10} expert}}
\vspace{.5cm}

\Qitem{ \Qq{Do you have a favorite distro?} \Qline{8cm} }

\Qitem{ \Qq{How long have you been using Linux?} \Qline{6cm}}

\Qitem{ \Qq{Any other topics, or any more specific ones?} \Qlines{4} }

\Qitem{ \Qq{Are there any topics you'd be willing to present on at a meeting?}
\Qlines{4} }

\end{document}
